\documentclass[../main/thesis_msc.tex]{subfiles}

\begin{document}
	\tableofcontents
	\newpage  % Optional - starts content on new page
    \chapter{Introduction}
    
	Drosophila melanogaster, the common fruit fly, has become an indispensable model organism in neuroscience research, particularly for studying behavior and neural circuits \autocite{bellen_100_2010} Its popularity stems from several key advantages: a short lifespan of about 80 days, rapid reproduction with females capable of laying up to 100 eggs daily, and a relatively simple nervous system of approximately 100,000 neurons compared to the human brain's 86 billion \autocite{hales_genetics_2015}. These characteristics make Drosophila an ideal subject for investigating fundamental neurobiological principles and conducting multi-generational studies.
	
	One fascinating aspect of Drosophila behavior is its flight pattern, which consists of straight flight segments punctuated by rapid turns known as body saccades. This flight strategy is believed to minimize motion blur, enabling flies to maintain visual acuity during flight. Drosophila can execute these saccades with remarkable agility, rotating their bodies by about 90° in just 50 ms, achieving peak angular velocities exceeding 1000 degrees per second \autocite{muijres_body_2015}.
	
	
	To study these impressive maneuvers, researchers employ two primary methods: tethered flight and free flight. Tethered flight experiments involve immobilizing flies on a holder while allowing their wings to move freely. This approach facilitates direct access to the nervous system through electrophysiology or calcium imaging and enables precise sensory stimulation \autocite{aymanns_descending_2022}. However, the constrained movement in tethered flight may not accurately reflect natural flight patterns. In contrast, free-flight apparatuses allow for more naturalistic behaviors but lack direct neuronal access, limiting the ability to record or modulate neuronal activity during flight \autocite{muijres_body_2015}.
	
	
	Using these methods, researchers have made significant strides in elucidating the neuronal mechanisms underlying control of behavior. Specific focus has been put on Drosophila's descending neurons (DNs) system. DNs comprise a group of approximately 1000-1300 neurons that connect the central nervous system (CNS) with the ventral nervous system (VNS) \autocite{azevedo_connectomic_2024,namiki_functional_2018}. These neurons play a crucial role in integrating and communicating multi-modal information from the CNS to the VNS, where they synapse directly with motor neurons or indirectly via interneurons. Recent studies have highlighted the importance of DNs in transforming sensory inputs and internal state information into motor outputs \autocite{namiki_functional_2018,simpson_descending_2024}.
	
	
	However, the precise mechanisms by which DNs control behavior remain a subject of ongoing research. Some studies suggest a "command-like" mechanism where a single DN can trigger a well-defined behavioral output, while others propose that complex behaviors require the coordinated activity of multiple DNs \autocite{cande_optogenetic_2018, braun2024descending} (braun 2024).
	
	
	Given this background, the primary aim of my research is to establish and test a system that allows for the study of how different DNs control or contribute to the control of flight maneuvers in free-flight conditions. To achieve this, I have designed and constructed a free-flight arena that enables real-time tracking of untethered flies combined with closed-loop activation of selected neurons using optogenetics. This system facilitates the extraction of general characteristics of free-flight behavior and allows for exploring other genetic manipulations, such as neuronal silencing, on flight performance.
	
	
	By focusing on several DNs known or suspected to be involved in saccade initiation and control, I intend to investigate whether the activation of a single DN is sufficient to elicit a complete saccade or if this behavior requires the coordinated activity of multiple DNs for full execution. This research aims to shed light on the neural mechanisms underlying the impressive flight maneuvers of Drosophila and contribute to our understanding of how complex behaviors are controlled at the neuronal level.
	
	
	\section{Flight of Drosophila}
	
	The flight of Drosophila melanogaster is characterized by two distinct phases: straight flight bouts interspersed with rapid, directional changes known as saccades \autocite{collett_visual_1975}. During straight flight, the fly maintains a stable velocity and heading, heavily reliant on visual feedback mechanisms. The optomotor response plays a pivotal role in this phase, stabilizing the flight path and adjusting flight speed in response to factors such as spatial and temporal frequency and the contrast of the visual stimulus. The optomotor response is modulated by the fly’s behavioral state, as the gain of motion-sensitive visual neurons is increased during walking or flying, and their visual tuning shifts towards a higher velocity \autocite{borst_fly_2010,jung_flight_2011,cellini2020hybrid}.
	
	Conversely, saccades represent abrupt and rapid turns that enable Drosophila to quickly alter their flight direction, allowing flies to sample different parts of the visual field, as well as minimize the time during which their visual system is compromised by motion blur \autocite{muijres_body_2015,reichardt1976visual, tammero_collision-avoidance_2002,wagner_flight_1997,wehrhahn_tracking_1982}. These quick maneuvers, typically spanning only a few wingbeats over a short time (around 40-60 ms), can be initiated either by specific visual cues like the expansion of an image on the fly's retina \autocite{censi_discriminating_2013,reiser_visual_2013, stewart_model_2010, tammero_collision-avoidance_2002}, or due to some internal stochastic process, that functions to optimize the fly’s search path \autocite{maye_order_2007,mongeau_drosophila_2017, reynolds_free-flight_2007}. Spontaneous saccades are highly stereotyped, consisting of a banked turn, optimized to maintain flight control and minimize retinal slip \autocite{muijres_body_2015}. On the other hand, escape saccades are initiated as an aversive response to a possible collision/predator and thus prioritize speed over stability, resulting in faster heading changes than spontaneous saccades \autocite{muijres_flies_2014}. While both escape and body saccades consist of banked turns, the dynamics of each maneuver are different enough to suggest that distinct motor programs produce them.

	\subsection{Looming stimulus}

	Looming stimuli are powerful visual cues that simulate objects approaching at a collision course, such as predators or obstacles. In Drosophila research, these stimuli have become a valuable tool for studying the neural circuits underlying threat detection and escape behaviors \autocite{devries_loom-sensitive_2012, fotowat_novel_2009}. Looming stimuli typically consist of expanding dark shapes on a light background, mimicking the visual effect of an approaching object, providing an ethologically relevant stimulus to probe how flies detect and respond to imminent collisions or threats in their environment \autocite{card_performance_2008, fotowat_novel_2009}.

	The critical characteristics of looming stimuli are their expanding size on the retina over time and the precise mathematical relationship that governs this expansion. This relationship is typically characterized by the ratio of the object's half-size ($\ell$) to its approach velocity ($v$), known as $\ell/|v|$ \autocite{fotowat_novel_2009, gabbiani_invariance_2001}. Larger $\ell/|v|$ values correspond to slower expanding stimuli, while smaller values indicate more rapid expansion. As the virtual object approaches, its angular size ($\theta$) on the retina increases nonlinearly, providing a complex but consistent visual cue for the fly's nervous system to process \autocite{devries_loom-sensitive_2012}.

	Processing looming stimuli in the Drosophila brain involves a sophisticated sequence of neural computations. Initial detection occurs in the retina, followed by processing through parallel ON and OFF motion detection pathways in the optic lobe \autocite{klapoetke_ultra-selective_2017}. These pathways, beginning with lamina monopolar cells L1 (ON) and L2 (OFF), diverge in the medulla before converging onto T4 and T5 cells in the lobula plate. T4 and T5 cells, which are direction-selective for bright and dark edge motion, respectively, provide inputs to lobula plate tangential cells (LPTCs). LPTCs integrate these local motion signals across their large dendritic arbors that span multiple columns in the lobula plate. This integration makes LPTCs sensitive to wide-field motion patterns and looming stimuli. By receiving inputs from many T4 and T5 cells tuned to different directions of motion, LPTCs can detect complex motion patterns like expanding motion characteristic of approaching objects, effectively acting as "matched filters" for behaviorally relevant visual motion. ON and OFF motion detection pathways detect brightness increments and decrements, respectively \autocite{behnia_processing_2014}. These pathways are not limited to looming detection but play a broader role in processing visual motion information. In the context of looming, both pathways contribute to detecting approaching objects, with the ON pathway being more sensitive to light objects on a dark background and the OFF pathway to dark objects on a light background. Importantly, LPTCs have been proposed to function as matched filters for complex optic flow patterns associated with a fly's movements. The response properties of several LPTCs have been characterized in detail, mainly through studies in larger flies. These cells show strongly directionally selective responses to a variety of motion stimuli. Recent advances have revealed key components of the upstream circuitry that provides LPTCs with their direction-selective response properties, furthering our understanding of visual motion processing in the fly brain.

	The behavioral response to looming stimuli in flies typically involves a well-coordinated escape sequence. This sequence begins with wing raising, followed by leg extension and take-off \autocite{card_performance_2008, fotowat_novel_2009}. The timing of these behaviors is tightly correlated with the angular size of the expanding stimulus, occurring at a fixed delay after the stimulus reaches a specific angular threshold. This allows flies to make split-second decisions about when to initiate their escape, balancing the need for a timely response against the risk of false alarms.

	One of the main descending neurons implicated in the behavioral response to looming stimuli is the giant fiber (GF) pathway \autocite{card_visually_2008}. The GF receives input from two key visual projection neurons (VPNs): LPLC2 and LC4 neurons. LPLC2 shows a high degree of selectivity for looming stimuli \autocite{ache_neural_2019,klapoetke_ultra-selective_2017}, and their primary role in the GF pathway is in encoding the angular size of the approaching object. LC4 works in concert with the LPLC2 neurons; however, it encodes the incoming stimulus's angular velocity rather than its size \autocite{ache_neural_2019}. The GF integrates these two distinct features to determine the timing of its spike, which in turn dictates the timing of the fly’s escape jump \autocite{ache_state-dependent_2019,von_reyn_spike-timing_2014, jang_azimuthal_2023}.

	\section{Anatomy}
	\subsection{Central Brain}
	The central brain of the fruit fly Drosophila melanogaster contains several key regions that integrate sensory information and coordinate complex behaviors through descending neurons that connect to motor circuits. Major areas include the mushroom bodies, which are critical for associative learning and memory; the central complex, involved in navigation, motor control, and sleep regulation; and the lateral horn, which processes innate olfactory behaviors. The lateral accessory lobe (LAL) and posterior slope (PS) are implicated in navigation and visual motion processing. In contrast, the posterior lateral and ventrolateral protocerebra (PLP, PVLP) are involved in escape and rapid responses to salient visual stimuli. The gnathal ganglion (GNG) plays a role in mechanosensory, gustatory, and locomotor responses \autocite{namiki_functional_2018}.

	The mushroom bodies receive olfactory, visual, and other sensory inputs, with their output neurons projecting to areas like the superior medial protocerebrum, which contains descending neurons that can influence locomotor behaviors. The central complex, consisting of interconnected neuropils, including the ellipsoid body, fan-shaped body, and protocerebral bridge, integrates visual and proprioceptive information to guide behaviors like walking and flight. Many central complex neurons project to regions like the lateral accessory lobe, which contains descending neurons involved in turning behaviors \autocite{scheffer_connectome_2020}.

	Descending neurons serve as the key link between these central brain regions and motor circuits in the ventral nerve cord, acting as a bottleneck for transmitting information from sensory and integrative brain areas to motor output circuits. Major classes of descending neurons originate in areas like the posterior slope, gnathal ganglia, and subesophageal zone. Those from the posterior slope tend to target wing neuropils and control aspects of flight, while those from the gnathal ganglia often innervate leg neuropils to modulate walking. Some DNs target neuropil layers containing wing, neck, and haltere motor neurons, while others innervate leg neuropils. A subset of DNs terminate in intermediate VNC layers thought to coordinate behaviors requiring both leg and wing control, like takeoff \autocite{azevedo_connectomic_2024,namiki_functional_2018,takemura_connectome_2023}.

	\subsection{Ventral Nerve Cord}

	The pathway connecting the central brain and the VNC can be generally separated into three pathways, each carrying information specific to a particular motor system \autocite{cheong_transforming_2024, namiki_functional_2018}. The first pathway consists of a significant number of DNs that connect the brain's inferior posterior slope (IPS) and superior posterior slope (SPS) to the dorsal neuropils of the VNC. This pathway is proposed to be involved in flight control, as the posterior slope receives input from lobula plate tangential cells (LPTCs) in the optic lobe, which respond to patterns of optic flow relevant to flight \autocite{borst_fly_2010, krapp_estimation_1996}. Additionally, the dorsal neuropils contain motor neurons for the neck, wing, and haltere and sensory neuron projections from related mechanoreceptors and premotor interneurons. The second pathway consists of DNs that connect the gnathal ganglia (GNG) to the ventral leg neuromeres of the VNC. This pathway is thought to be evolutionarily ancient and potentially homologous with pathways in other arthropod taxa \autocite{niven_diversity_2008}. This pathway involves many behaviors, such as walking, flight initiation, head movement, and respiration \autocite{cande_optogenetic_2018, namiki_functional_2018}. However, it has been suggested as important for the temporal patterning of walking \autocite{gal_new_2006}. The third is a convergent pathway in which DNs originating from various brain neuropils, except for the mushroom bodies and central complex, converge to innervate the tectulum. The tectulum is situated between the dorsal motor centers and leg neuromeres, spanning all three thoracic segments. Given its location and diverse inputs, the tectulum is thought to function in behaviors requiring significant sensory integration and motor coordination, such as courtship or grooming \autocite{namiki_functional_2018}.

	In addition to these three main pathways, a small population of DNs project to the lower tectulum, a region between the tectulum and leg neuropils. This region may be involved in coordinating leg and wing movements, particularly during takeoff, as it contains dendrites from leg and wing motor neurons, as well as the peripheral synapsing interneuron (PSI) and the terminals of the giant fiber (GF) descending neurons \autocite{card_escape_2012,von_reyn_spike-timing_2014}.

	This anatomical separation may reflect the functional organization of the DNs and the evolutionary history of Drosophila. Insects evolved flight differently than other flying animals, whose wings are modified forelegs \autocite{gatesy_locomotor_1996}. This difference in evolution is manifested in the organization of the VNC and the pattern of DN innervation observed in Drosophila. The leg neuromeres resemble those of their ancestors, while the wing neuropil, a more recent evolutionary development, sits atop the VNC. The organization of DNs suggests a hierarchical flow of information from the brain to the VNC. 

	\subsection{Descending Neurons}

	Previous studies estimate the number of DNs to be around 1000, which is several orders of magnitude less than the number of neurons in the brain \autocite{hsu_organization_2016, namiki_functional_2018}. This system thus represents a critical bottleneck in transmitting information from the sensory processing centers in the brain and the pre-motor circuits in the ventral nerve cord (VNC), as complex brain activity must be compressed into simpler commands that can be effectively interpreted and executed by motor circuits.

	DNs receive input from various sensory systems, including visual, olfactory, and mechanosensory systems, such as the optic glomeruli and the AMMC (antennal mechanosensory and motor center); and higher-order brain regions, such as the lateral accessory lobe (LAL), which is densely interconnected with the central complex. This input structure allows them to integrate sensory information and generate appropriate motor responses. DNs also express a variety of neurotransmitters, including acetylcholine, GABA, glutamate, serotonin, dopamine, and octopamine. These neurotransmitters can modulate the activity of neurons in the VNC, thereby fine-tuning behavior \autocite{hsu_organization_2016, namiki_functional_2018}.

	\subsection{DNs Involved in Flight Control}

	The control of flight by DNs in Drosophila involves a complex interplay between descending commands from the brain and local circuits within the VNC. DNs project from the brain to the VNC, interacting with interneurons and motor neurons to generate coordinated flight behaviors. The MANC connectome has revealed that DNs targeting the wing neuropils are organized into distinct functional groups, with some primarily innervating the wing tectulum (WTct) and others targeting both the wing and haltere tectula \autocite{cheong_transforming_2024}. These DNs exhibit varying degrees of bilateral and unilateral connectivity, suggesting roles in both symmetric and asymmetric flight control.

	Several descending neurons have previously been described as correlated with saccadic flight turns. 

	\subsubsection{AX Neuron}

	The AX neuron, a descending neuron correlated with rapid fly turns in tethered Drosophila, provides insight into how DNs control steering maneuvers \autocite{schnell_descending_2017}. Two-photon calcium imaging revealed that unilateral AX activity precedes ipsilateral turning behavior, with the magnitude of the turn correlating with the amplitude of the AX response. This relationship holds true for both spontaneous and looming-elicited saccades, suggesting that AX is not solely relaying sensory information but may also be involved in the internal generation of turning behavior. Furthermore, focal application of ATP to P2X2-expressing DNaX neurons causes both increased neural activity and ipsilateral turning, indicating that activation of this neuron alone may be sufficient to elicit a turning response.

	\subsubsection{DNae014 and DNb01}

	Recent work has identified another descending neuron, DNae014, that shows tight correlation with saccadic turns in tethered flies \autocite{ros_descending_2024}, complementing the earlier findings about the AX neuron \autocite{schnell_descending_2017}. Two-photon calcium imaging was used to examine the activity of DNae014 during flight. The researchers found that DNae014 activity was very tightly coupled with turning behavior, with nearly every saccade accompanied by a corresponding change in DNae014 activity. This strong correlation between DNae014 activity and turning was observed without visual stimuli, further supporting the neuron's role in spontaneous turning behavior. To test whether DNae014 activity is sufficient to elicit a turn, the researchers used both focal two-photon excitation and the SPARC method to activate DNae014 neurons expressing CsChrimson unilaterally. Both methods yielded similar results: activating the right DNae014 neuron reliably triggered a rightward turn, while activating the left neuron triggered a leftward turn. These results suggest that DNae014 activity alone is sufficient to trigger a saccade. Finally, to assess the necessity of DNae014 activity for saccade generation, Ros et al. silenced the neuron using both optogenetic silencing and genetic ablation. Optogenetic silencing did reduce the frequency of saccades, but this effect was difficult to interpret due to a simultaneous increase in saccade rate in control flies exposed to the activation light. Genetic ablation of DNae014, however, produced a clearer result, significantly reducing the frequency of saccades compared to control flies. Interestingly, while ablating DNae014 significantly reduced the frequency of saccadic turns, it did not completely abolish them. This suggests that other neural pathways may also contribute to saccade generation.

	The DNae014 neuron is part of a network of descending neurons (DNs) that control saccades. This network includes another DN, DNb01, whose activity correlates with saccades. The two neurons appear to form a functional unit, which the researchers termed a saccade generating unit (SGU), with one unit for rightward turns and one for leftward turns. The DNae014 neuron within an SGU makes ipsilateral connections within both the brain and ventral nerve cord (VNC), meaning that it connects to neurons on the same side of the body. In contrast, the DNb01 neuron makes contralateral connections to the ventral nerve cord and has a more complex pattern of connections within the brain. This difference in connectivity, along with other evidence, suggests that the two neurons may drive different elements of the saccade motor program within the VNC.

	Additionally, the researchers found that VES041 neurons - a pair of large inhibitory interneurons in the brains of fruit flies - play a crucial role in regulating flight turns, specifically by suppressing spontaneous saccades. Located in the fly brain's lateral accessory lobe (LAL), these neurons achieve this suppression by forming inhibitory connections with the DNae014 and DNb01 neurons. By inhibiting these saccade-generating units (SGUs), VES041 neurons promote straight flight. This suggests that VES041 neurons are likely involved in regulating transitions between different flight behaviors, such as shifting from local exploration, characterized by frequent turns, to long-distance dispersal, which necessitates straighter flight paths. VES041 neurons receive input from various brain regions, including those associated with visual and olfactory processing, such as the vest, GNG, LAL, SPS, flange, inferior bridge, and saddle. This diverse input suggests that VES041 neurons integrate information from multiple sensory modalities to modulate turning behavior in response to environmental cues.

	\subsubsection{DNp06}

	Another descending neuron, DNp06, has been identified as potentially involved in the control or initiation of saccadic turns \autocite{kim_visuomotor_2023}. DNp06 differs from the previously described DNs in that it receives visual input from LPLC2 neurons, allowing it to integrate multiple visual features from the lobula and lobula plate. The integration of these signals allows the LPLC2 neurons to detect visual features associated with imminent threats, such as approaching or rapidly moving objects. This information is then relayed to the DNp06 neurons, which likely activate wing-steering muscles to execute an evasive turn. Silencing DNp06 neurons diminishes the flies' wing responses to both translating spots and looming discs, indicating a reduced capacity for visually guided avoidance behavior. Conversely, direct activation of these neurons using optogenetic techniques reliably triggers flight turns, demonstrating their sufficiency in eliciting such maneuvers. Electrophysiological recordings from DNp06 neurons reveal a strong, phasic response to visual stimuli, such as moving bars and spots, presented to the contralateral visual field. This suggests that DNp06 neurons primarily receive visual information from the opposite side of their location in the brain, consistent with their role in orchestrating turns away from threats in that visual field.

	\section{Genetic Tools}
	\subsection{GAL4-UAS System}

	The GAL4-UAS system has become an indispensable tool for targeted gene expression in Drosophila neuroscience research. This binary expression system was first adapted for use in fruit flies from yeast \autocite{brand_targeted_1993}. The system consists of a "driver" line expressing the transcriptional activator GAL4 under the control of a specific promoter or enhancer, and a "responder" line carrying a gene of interest under UAS (upstream activating sequence) control. When these lines are crossed, GAL4 binds to the UAS and activates expression of the responder gene in cells where the driver is active. The power of the GAL4-UAS system lies in its versatility and modularity. Thousands of GAL4 driver lines have been generated, each with a distinct expression pattern, allowing researchers to target gene expression to specific neurons or cell types of interest. When combined with the wide array of UAS-linked effector genes (e.g., fluorescent proteins, calcium indicators, optogenetic tools), this provides exquisite control over labeling, monitoring, and manipulating defined neuronal populations.

	Over the years, several refinements have enhanced the utility of the GAL4 system. The addition of temperature-sensitive GAL80 allows temporal control of GAL4 activity \autocite{mcguire_spatiotemporal_2004}.

	\subsection{Split-GAL4 System}

	The split-GAL4 system emerged as a refinement of the original GAL4-UAS system, building on the discovery that the DNA-binding and transcriptional activation functions of GAL4 could be separated into distinct protein domains \autocite{keegan_separation_1986,ma_deletion_1987}.

	In the split-GAL4 system, the GAL4 DNA-binding domain (DBD) and activation domain (AD) are expressed as separate proteins under the control of different promoters. Only in cells where both promoters are active will the two parts come together, reconstituting a functional GAL4 transcription factor. This reconstituted GAL4 can bind upstream activating sequence (UAS) sites and drive the expression of UAS-controlled transgenes. By using two different promoters to drive the DBD and AD, greater specificity of transgene expression can be achieved compared to using either promoter alone \autocite{luan_refined_2006,pfeiffer_refinement_2010}.

	The main advantage of split-GAL4 is the ability to refine and restrict transgene expression patterns through intersectional strategies. This allows the targeting of more precise subsets of cells than traditional GAL4 drivers. A disadvantage initially was lower expression levels compared to intact GAL4 \autocite{luan_refined_2006}. However, this has been improved by optimizing the activation domains, such as replacing the original GAL4 activation domain with stronger alternatives like VP16 or p65 \autocite{pfeiffer_refinement_2010}.

	Using the Split-GAL4 system, the FlyLight Project Team at the Janelia Research Campus has generated thousands of driver lines targeting small subsets of neurons throughout the Drosophila nervous system \autocite{dionne_genetic_2018}. By first identifying promising enhancer fragments that drive expression in neurons of interest using conventional GAL4 lines, they created Split-GAL4 hemidrivers (lines expressing either the DBD or AD component) for these enhancers. By systematically testing combinations of these hemidrivers, they can often achieve highly specific expression in small groups of neurons or even single cell types.

	\subsection{Stochastic Experssion Tools}
	\subsubsection{Heatshock-dependent Flippase (FLP)}

	Relying on stochastic expression systems, such as the heat-shock FlpStp and SPARC systems, can further improve the specificity of these tools. The Flp-FRT system, pioneered in Drosophila \autocite{golic_flp_1989, struhl_organizing_1993}, has emerged as a powerful tool for achieving both spatial and temporal control over gene expression. This system combines a heat-shock inducible Flippase (Flp) recombinase with FRT site-flanked transgenes, allowing researchers to induce genetic changes at specific times and in specific tissues. The key components include a heat-shock promoter (typically hsp70 or hsp26) driving the Flp recombinase gene, a transgene of interest flanked by FRT sites (often with a stop cassette between the promoter and coding sequence), and a tissue-specific or ubiquitous promoter driving expression of the FRT-flanked transgene \autocite{mcguire_spatiotemporal_2004}. At room temperature, under normal conditions, the stop cassette prevents the expression of the transgene. However, when a heat shock is applied (typically at 37°C for 30-60 minutes), the heat-shock promoter is activated, leading to the expression of the Flp recombinase. The Flp recombinase mediates stochastic recombination between the FRT sites, excising the stop cassette and allowing transgene expression \autocite{golic_flp_1989}.

	The Flp-FRT system offers precise temporal control, as transgene expression can be induced at any developmental stage simply by applying a heat shock \autocite{struhl_organizing_1993}. Spatial control can be achieved by using tissue-specific promoters to drive the FRT-flanked transgene, and it can be combined with other expression systems like GAL4-UAS for additional experimental flexibility \autocite{brand_targeted_1993, mcguire_spatiotemporal_2004}. However, there is often some basal Flp expression even without heat shock, which can lead to leaky expression of the transgene. Additionally, the kinetics of transgene induction can vary depending on the specific transgene and may take hours to days to reach full expression \autocite{mcguire_spatiotemporal_2004, struhl_organizing_1993}.
	\subsubsection{Sparse Predictive Activity through Recombinase Competition (SPARC)}
	The SPARC (Sparse Predictive Activity through Recombinase Competition) method is a technique for achieving targeted expression of user-selected genes in precise proportions of cells in Drosophila \autocite{isaacman-beck_sparc_2020}. This method leverages the stochastic nature of PhiC31 recombinase activity to achieve different levels of gene expression in a predictable manner \autocite{monetti_phic31_2011}. Notably, the SPARC method can be used to express genes in a sparse, predictable manner, which has yet to be possible with previous techniques. 

	A genetic construct contains a user-selected effector gene downstream of a stop cassette flanked by two attP recombination sites recognized by PhiC31. One of these attP sites is a canonical (i.e., unmodified) sequence, while the other is truncated. The construct is designed such that PhiC31-mediated recombination between the first attP sequence and a separate attB sequence will excise the stop cassette, enabling effector expression. However, if PhiC31 recombines the second attP site with the attB site, the stop cassette is retained, and effector expression is prevented. By changing the length of the truncated attP site, it is possible to control the probability of each recombination event precisely.  This is because truncated attP sites have a lower recombination efficiency with attB sites. Three SPARC variants are available with different attP truncations: D (Dense), I (Intermediate), and S (Sparse). As their names suggest, these variants are expected to produce high, intermediate, and low levels of effector expression, respectively. Importantly, SPARC consistently labeled similar percentages of neurons across a diverse array of cell types. Specifically, the SPARC-D, -I, and -S variants consistently labeled ~48-51\%, ~17-22\%, and ~3-7\% of cells, respectively.

	\subsection{Manipulation of Neuronal Activity}
	\subsubsection{Tools for Inhibition of Neurons}

	Genetic tools for manipulating neuronal activity have become invaluable in Drosophila neuroscience research. Two main approaches have emerged as particularly powerful: overexpression of specific ion channels, such as the potassium channel Kir2.1, to electrically inactivate neurons; and expression of neurotoxins like tetanus toxin light chain (TeTxLC) to inhibit synaptic transmission. Researchers can study the effects on neural circuits and behavior by employing these tools to target specific neurons.

	The Kir2.1 channel is an inward-rectifying potassium (K+) channel that plays a role in electrically inactivating neurons by causing hyperpolarization. When expressed in Drosophila neurons, Kir2.1 increases potassium efflux, lowering the resting membrane potential below the threshold needed to fire action potentials, effectively silencing the neuron. This makes Kir2.1 a powerful tool for manipulating neuronal activity in targeted neural circuits to study behaviors and phenotypes in flies. The advantages of using Kir2.1 include its ability to strongly inhibit neuronal activity and facilitate precise functional mapping of neural circuits. However, this approach has potential drawbacks, including induction of compensatory homeostatic mechanisms and risk of apoptosis or lethality in essential neurons \autocite{hodge_ion_2009}. An alternative approach is the tetanus toxin light chain system \autocite{sweeney_targeted_1995}, which blocks neurotransmitter release by cleaving the synaptic vesicle protein synaptobrevin. TeTxLC offers advantages in its specificity for neuronal synaptobrevin and cell-autonomous action. When expressed neuronally, it generally avoids affecting non-neuronal tissues or causing obvious developmental defects. However, it may not completely block neuropeptide release from dense core vesicles \autocite{kaneko_disruption_2000}.

	Both systems can be expressed using the GAL4/UAS binary system for spatial and temporal control. This allows the silencing of specific neuronal populations by crossing GAL4 driver lines to UAS-Kir2.1 or UAS-TeTxLC responder lines. An important consideration is that many GAL4 lines have broad expression patterns, so observed phenotypes may result from silencing multiple neuron types of neurons.

	\subsubsection{Optogenetics}

	Optogenetics is a technique that enables neural activity control using light-sensitive proteins called opsins. The field developed in the early 2000s following research into light-sensitive microbial proteins. In 2005, channelrhodopsin-2 (ChR2), an ion channel from algae, was demonstrated to allow light-based control of mammalian neurons \autocite{boyden_millisecond-timescale_2005}. ChR2 opens in response to blue light, permitting cation influx and subsequent action potential generation. This method allows for temporal control of specific neural populations.

	Subsequently, additional opsins have been identified or engineered, expanding the available tools. These include inhibitory opsins such as halorhodopsin, which imports chloride ions to suppress neural activity, and red-shifted variants activated by longer wavelengths \autocite{inagaki_optogenetic_2014}. The optogenetic process involves the genetic modification of neurons to express opsins, light delivery to the brain, and activation or inhibition of opsin-expressing neurons using specific light wavelengths.

	The adaptation of optogenetics for Drosophila research encountered several obstacles. Adult fly cuticle poorly transmits blue light, limiting ChR2 efficacy. Furthermore, the fly visual system's sensitivity to blue light introduced behavioral artifacts in experiments \autocite{kim_cellular_2015}. These issues were addressed through the development of red-shifted opsins like ReaChR and CsChrimson, which are activated by red light that more effectively penetrates tissue \autocite{inagaki_optogenetic_2014,klapoetke2014independent} Improved genetic expression systems increased opsin levels in neurons. Researchers also developed methods to use lower light intensities and, in some cases, block visual inputs \autocite{kim_cellular_2015}. Additionally, unlike mammals, Drosophila do not naturally produce sufficient amounts of all-trans-retinal, the chromophore required for opsin function. Therefore, flies typically require dietary supplementation with all-trans-retinal for effective optogenetic experiments \autocite{kim_cellular_2015}.

	\section{Methods to Track Flying Insects}

	Research on Drosophila flight behavior, particularly saccades, has employed two primary methods: tethered preparations and free-flight paradigms. Early tethered flight studies \autocite{gotz_flight_1968, vogel_i_1966} allowed precise measurement of flight forces and wing kinematics. However, saccadic behavior in tethered flight differs significantly from free flight. In rigidly tethered flies, saccades last approximately 500 ms, about ten times longer than the 40-60 ms duration observed in free flight \autocite{bender_comparison_2006, frye_odor_2003}. This duration discrepancy stems primarily from the absence of haltere feedback in rigidly tethered conditions \autocite{bender_visual_2006, frye_motor_2004}. The halteres, modified hindwings that act as gyroscopes, provide rapid mechanosensory feedback during rotation. When flies are tethered in a way that allows free yaw rotation, such as with a magnetic tether, saccade durations more closely match free flight, typically lasting 60-90 ms \autocite{bender_comparison_2006}.

	However, researchers recognized the limitations of tethered preparations and the need to study flight behavior in freely moving flies. In 1978, David et al. published one of Drosophila's first quantitative studies of free-flight kinematics, using a wind tunnel and manual observations to measure body angle and flight speed. This provided important data on how flies modulate their body posture during flight, but it needed to be improved in capturing detailed 3D trajectories.

	Over the following decades, 3D tracking systems for Drosophila underwent significant advancements driven by key technological improvements. Camera technology progressed rapidly, allowing for higher frame rates and improved resolution, which enabled more precise capture of the flies' rapid movements. Concurrently, computer vision algorithms became more advanced, enhancing the ability to detect and track flies accurately within complex visual scenes. These systems evolved to handle multiple flies simultaneously, opening up new possibilities for studying social interactions and group behaviors.

	Modern tracking systems can accurately track multiple flies' 3D position and orientation in real-time \autocite{straw_multi-camera_2011}. These systems use careful camera calibration to determine relative positions, orientations, and intrinsic parameters. During experiments, each camera's images are processed in real-time using computer vision algorithms, often employing background subtraction to extract 2D coordinates \autocite{fry_aerodynamics_2005}.

	The 2D coordinates from multiple camera views are combined through triangulation to estimate the insect's 3D position. This process finds the 3D point that best satisfies the geometric constraints from each camera's 2D observations \autocite{straw_multi-camera_2011}. A Kalman filter is applied to handle measurement noise and potential occlusions. This recursive Bayesian estimation technique uses a motion model to predict the insect's state and corrects this prediction based on observed 3D positions, providing a statistically optimal estimate of the insect's position and velocity \autocite{straw_multi-camera_2011}.

	Many systems are designed to operate in real-time, performing triangulation and updating the Kalman filter estimate multiple times per second. This real-time processing enables closed-loop experiments where the visual environment can be dynamically updated based on the insect's position. Through this combination of computer vision, multi-view geometry, and probabilistic filtering, researchers can achieve precise 3D tracking of fast-moving insects across large volumes, providing invaluable data for neuroscience studies of flight behavior.


\end{document}
